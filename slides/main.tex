\documentclass{beamer}
\usepackage[utf8]{inputenc}
\usepackage{beamerthemebars}


\usetheme{Madrid}
\usecolortheme{default}

%------------------------------------------------------------
%This block of code defines the information to appear in the
%Title page
\title[] %optional
{Franken-LLAMA: Experimenting with LLama2 surgery to make it more efficient}

\subtitle{Project Work in APAI}

\author[Angelo Galavotti] % (optional)
{Presented by Angelo Galavotti}

\date[December 2024] % (optional)
{Università di Bologna, December 2024}

%End of title page configuration block
%------------------------------------------------------------



%------------------------------------------------------------
%The next block of commands puts the table of contents at the 
%beginning of each section and highlights the current section:

\begin{document}

%The next statement creates the title page.
\frame{\titlepage}


%---------------------------------------------------------
%This block of code is for the table of contents after
%the title page


\section{Introduction}
\begin{frame}
  \frametitle{Quick overview on LLaMA 2}
  LLaMA 2 (Large Language Model Meta AI version 2) is an \textbf{open-source} family 
  of foundational language models created by Meta.
  \begin{itemize}
    \item Available in \textbf{7B, 13B, and 70B} (billion parameters) versions.
    \item Performance is comparable to OpenAI's ChatGPT 3.5.
    \item Comes in a \textbf{base version} and \textbf{chat version}, fine-tuned with
    \textbf{RLHF} (Reinforcement Learning with Human Feedback) for conversational use.
  \end{itemize}
  \end{frame}

%---------------------------------------------------------

\section{Second section}

%---------------------------------------------------------
%Highlighting text
\begin{frame}
\frametitle{Franken-LLAMA}
\textbf{Franken-LLAMA} consists in optimizing LLaMA 2 (7B Chat) by reducing 
computational and memory costs through \textbf{layer skipping} and \textbf{repetition}, effectively performing "surgery" on the model.
\begin{figure}[h] % 'h' places the figure approximately here
  \centering
  \includegraphics[width=0.3\textwidth]{images/image.png}
\end{figure}
\end{frame}

\begin{frame}
  \frametitle{Configurations}
  A total of \textbf{25 configurations} of skipped and repeated layers were tested by completing the phrase "Once upon a time" using a maximum of 50 tokens.

  Most of them produced gibberish text.
\end{frame}

\begin{frame}
  \frametitle{Output examples}
  \begin{table}[h]
    \centering
    \begin{tabular}{|l|p{0.58\textwidth}|}
        \hline
        \textbf{Configuration} & \textbf{Generated text} \\ \hline
        \small \url{0-7} & {\small Once upon a timezetempreasacondarichte?? trickster goddess pue moonkennecticut [..] Reserveikaiwitzter PetersburgovPortail [..]} \\ \hline
        \small \url{all_except_last_two} & {\small Once upon a time year0 **stadt [..] Death it Yearwaltapk Progress R?f?rencePU. ??? [..]} \\ \hline
        \small \url{only_even_layers} & {\small Once upon a time S??. R S l d d? S S S S S [..]} \\ \hline
        \small \url{0r2-14_17-23_25-31r2} & {\small Once upon a time, in the midst of a busy schedulesomeone's attention was caught.?You the world and its of the, and and and [..]} \\ \hline
        \small \url{first_last_8r2} &{\small Once upon a time in?ceycofortia-inaymskoe Bridge---Monlinaiticky'830 [..]} \\ \hline
        \small \url{15r3_23r3_31r3} &{\small Once upon a timepus pri rosgemeingemeingemeinwach junigemeingemei [..]} \\ \hline
      \end{tabular}
    \label{tab:generated_text}
  \end{table}
\end{frame}


\begin{frame}
  \frametitle{Configurations}
  After careful examination, only \textbf{6 configurations} were ultimately chosen for more thorough testing,
  \begin{itemize}
    \item \url{'0-23\_27-31'}, skips layers from 24 to 26
    \item \url{'15\_single\_skip'}, skips only layer 15
    \item \url{'mid\_expansion\_with\_repeats'}, skips layers [6, 7, 8, 9, 25, 26, 27, 28], repeats layers from 14 to 19 twice
    \item \url{'2\_3rd\_of\_llama'}, skips layers from 11 to 20
    \item \url{'2\_3rds\_plus\_llama'}, skips only odd indexed layers from 11 to 20
    \item \url{'skip\_near\_end\_keep\_last\_two'}, skips layer from 27 to 29
  \end{itemize}
  The least computationally costly configuration is \url{'2\_3rd\_of\_llama'} ($\sim 6.2$M MACs)
  \begin{itemize}
    \item even though it doesn't involve layer repetition, it is also the \textbf{lightest memory-wise}.
    \item however, it also produced the \textbf{lowest quality} outputs.
  \end{itemize}
  \end{frame}

\begin{frame}
  \frametitle{Experimental Setup}
  The 6 configuration tested on the \textbf{HellaSwag} dataset and were compared
  to a \textbf{baseline} (i.e. the full Llama1-7B-chat model).
  \begin{itemize}
    \item The HellaSwag dataset consists on 4 questions with only one right answer. Each model has to predict the right answer. It allows to test the \textbf{logical capabilities} of a model.
    \item The models were evaluated on 50 samples of this dataset.
    \item Each model has also undergone a preliminary qualitative test by generating answers to 50 samples of the Natural Questions dataset.
  \end{itemize}
\end{frame}


\begin{frame}
  \frametitle{Results on HellaSwag}
  \begin{table}[h]
    \centering
    \begin{tabular}{|l|c|c|}
        \hline
        \textbf{Configuration} & \textbf{HellaSwag} & \textbf{Avg. exec. time} \\ \hline
        \url{baseline} & 0.34 & 91.1 s\\ \hline
        \url{0-23_27-31} & 0.38 & 81.2 s\\ \hline
        \url{15_single_skip} & 0.38 & 95.8 s\\ \hline
        \url{mid_expansion_with_repeats} & 0.22 & \textbf{68.8 s} \\ \hline
        \url{2_3rd_of_llama} & 0.26 & 95.4 s\\ \hline
        \url{2_3rds_plus_llama} & 0.30 & 102.7 s\\ \hline
        \url{skip_near_end_keep_last_two} & \textbf{0.42} & 79.0 s\\ \hline
    \end{tabular}
    \caption{HellaSwag scores (i.e. percentage of correct answers)}
    \label{tab:model_comparison}
\end{table}
\end{frame}

%---------------------------------------------------------

\begin{frame}
  \frametitle{In conclusion}
  \begin{enumerate}
    \item Skipping and repeating layers can be a \textbf{viable solution} to make light-weight LLMs.
    \item Generally, repetition can cause the \textbf{quality of the result} to \textbf{drop} significantly.
    \item Best results are achieved when skipping layers in the \textbf{middle} of the feature extractor.
  \end{enumerate}
  Future work:
  \begin{itemize}
    \item Testing with more samples/configurations.
    \item kV cache compression to reduce memory footprint even further.
  \end{itemize}
\end{frame}
%---------------------------------------------------------
\begin{frame}

  \begin{figure}[t]
    \includegraphics[width=8cm]{images/ending.png}
    \centering
    \end{figure}

    \vspace{0.2cm}


  \begin{center}
    \begin{Large}
      Thank you!
    \end{Large}
    \\ You can reach out at \url{angelo.galavotti@studio.unibo.it} for more questions!
  \end{center}
\end{frame}
%---------------------------------------------------------


\end{document}